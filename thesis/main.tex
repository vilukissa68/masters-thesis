\documentclass[12pt,a4paper,english
% ,twoside,openright
]{tunithesis}

% Note that you must choose either Finnish or English here and there in this
% file.
% Other options for document class
  % ,twoside,openright   % If printing on both sides (>80 pages)
  % ,twocolumn           % Can be used in lab reports, not in theses

% Ensure the correct Pdf size (not needed in all environments)
\special{papersize=210mm,297mm}


% LaTeX file for BSC/MSc theses and lab reports.
% Requires the class file (=template) tunithesis.cls and figure files,
% either tut-logo, exampleFig (as pdf or eps) and example_code.c
% Author: Lucas Machado (2018)
% Based on TTU template by Sami Paavilainen (2006), modified by Heikki Huttunen (2014)

% More information about Latex basics:
% [Tobias Oetiker, Hubert Partl, Irene Hyna, Elisabeth Schlegl, The
% Not So Short Introduction to LATEX2e, Version 5.03, April 2014, 171
% pages.  Availbale: http://tobi.oetiker.ch/lshort/lshort.pdf]

\author{Forename Surname}
\title{Thesis title} % primary title (for front page)
\thesistype{Master's thesis} % or Bachelor of Science, Laboratory Report... 

% Put your thesis' main language last
% http://mirrors.ctan.org/macros/latex/required/babel/base/babel.pdf
\usepackage{lastpage}
\usepackage[english]{babel}
\usepackage[
backend=biber,
style=authoryear,
citestyle=authoryear,
autocite=inline
]{biblatex}
\usepackage{csquotes}

\addbibresource{thesis_refs.bib} %Imports bibliography file

\definecolor{tunipurple}{RGB}{78, 0, 142}

\newcommand\todo[1]{{\color{red}!!!TODO: #1}} % Remark text in braces appears in red
\newcommand{\angs}{\textsl{\AA}}              % , e.g. slanted symbol for Ångstöm

\pagenumbering{roman} % was: {Roman}
\pagestyle{headings}
\begin{document}

% Special trick so that internal macros (denoted with @ in their name)
% can be used outside the cls file (e.g. \@author)
\makeatletter

\thispagestyle{empty}
\vspace*{-.5cm}\noindent

\begin{figure}
    \vspace{-1.3cm}
    \advance\leftskip-2.5cm
    \noindent\includegraphics{img/tunilogo.png}
\end{figure}
 
\vspace{2.5cm}
\begin{flushright}
\noindent\textsf{\LARGE{\@author}}

\noindent\vspace{0.5cm}

\noindent\Huge{\textsf{\textbf{\textcolor{tunipurple}{\@title}}}}
\end{flushright}
\vspace{13.7cm} % adjust to 12.7 this if thesis title needs two lines

% Last some additional info to the bottom-right corner
\begin{flushright}  
    \begin{spacing}{1.0}
      \textsf{Faculty of Information Technology and Communication Sciences (ITC)\\
      \@thesistype\\
      December 2024}
    \end{spacing}
\end{flushright}

% Leave the backside of title page empty in twoside mode
\if@twoside
\clearpage
\fi

% Turn off page numbering for the first pages
\pagenumbering{gobble}

\chapter*{Abstract}

\begin{spacing}{1.0}
\noindent \@author: \@title\\
\@thesistype\\
Tampere University\\
Master’s Degree Programme in Signal Processing and Machine Learning\\
December 2024
\end{spacing}
\noindent\rule{12cm}{0.4pt}

\vspace{0.5cm}

% ---------------------------------------
% Abstract and keywords
% ---------------------------------------

\noindent Lorem ipsum~

\noindent\textbf{Keywords:} DLA, Deep-Learning, SoC, Virtual Prototype.

~

\noindent The originality of this thesis has been checked using the Turnitin Originality Check service.


\setcounter{tocdepth}{3}              % How many header level are included
\tableofcontents                      % Create TOC


% The actual text begins here and page numbering changes to 1,2...
% Leave the backside of title empty in twoside mode
\if@twoside
%\newpage
\cleardoublepage
\fi


\renewcommand{\chaptername}{} % This disables the prefix 'Chapter' or
                              % 'Luku' in page headers (in 'twoside'
                              % mode)


\chapter{Introduction}
\label{ch:introduction}
\pagenumbering{arabic}
\setcounter{page}{1} % Start numbering from zero because command
                     % 'chapter*' does page break

In recent years neural network based application have become more and more prominent in our everyday-life. The large driver for this has been the adoption of efficient accelerators in mobiled device, that have enabled running neural network applications of mobile devices, such as smart phones.

The goal of this project was to build software support for the Deep-Learning Accelerator in the upcoming Headsail SoC from SocHub using a Renode based virtual prototype as the development board.

\chapter{Background}
\label{ch:background}

\section{Deep-learning accelerators}
\label{sec:dlas}
In desktop applications and data center workloads neural networks have been accelerated with GPUs, due to their ability to perform linear-algebra operations like matrix multiplication with high amount of parallellity.

In recent times there has been a growing need to also run neural networks in mobile devices. Traditional GPUs are too power-hungry to efficiently accelerate these networks, so a need for more efficient accelerators for these workloads has been born. Companies such as Apple and Qualcomm now include multiple mobile DLA's in their SoCs to run applications like face recognition on their phones.

\section{Headsail}
Headsail is the third Soc build by the SocHub research group. Headsail has two RISCV CPUs, one 32-bit meant for booting up the system called Sysctrl and one 64-bit on called HPC, meant for running the actual applications.
Headsail includes a wide variety of different peripherals, one of which is a custom build the Deep Learning accelerator.

\chapter{Methodology}
\label{ch:methodology}

\section{Renode}
\label{sec:renode}
Renode is a software development framework, which enables developers to use principles of continuous integration when writing hardware dependent code. In essence Renode is a hardware emulator which allows the user to specify exactly which kind of hardware they want to target, down to the implementation of specific peripherals and memory addresses. This streamlines the process of HW/SW integration, since hardware and software can be developed in parallel, which in return reduces the total production time for products.

Renode models a wide variety of different processors and peripherals, but it also expandable with custom components that are either baked directly into the binary (source code extensions in C\#) or with dynamicly loaded python peripherals. Python peripherals are more limited when compared to the C\# peripherals. This project implements the DLA hardware design as a dynamic python peripheral.

\chapter{Implementation}
\label{ch:implementation}

\section{Software support}
\label{sec:software_support}
Even though Headsail is the third SoCHub Soc, it had very little existing software support for C. Previous SoCs had only support for Riscv-rt in rust. So a major part of this project involved setting up a Headsail compatible C toolchain. Since Headsail uses includes RISCV CPUs we could use an already existing riscv-gnu-toolchain for the compiler, but we still had to set up a C standard library for the chip with custom version of newlib libgloss. Also due to specific memory addressing decisions in the hardware, we needed to use medany code model compatible compiler and standard library when targetting the 64-bit processor.

\chapter{Conclusions}
\label{ch:conclusions}

%
% The bibliography, i.e the list of references
%
\newpage

\printbibliography[title=References]
\addcontentsline{toc}{chapter}{References}


%
% Appendices are optional. 
% This part is semi-ugly at the moment. Please give feedback if can
% improve it.

\appendix
\pagestyle{headings}



%
% a) Not-so-handy way, but at least it works
% 
\def\appA{APPENDIX A. Something extra} % Define the name and numbering manually
\chapter*{\appA}                       % Create chapter heading
\markboth{\appA}{\appA}                % Set page header
\addcontentsline{toc}{chapter}{\appA}  % Include this in TOC
% Note that \label does not work with unnumbered chapter

Appendices are purely optional.  All appendices must be referred to in
the body text

\def\appB{APPENDIX B. Something completely different} % Define another new command
\chapter*{\appB}                       % As above, but use \appB instead of \appA
\label{app:B}
\markboth{\appB}{\appB}                     
\addcontentsline{toc}{chapter}{\appB}  


You can append to your thesis, for example, lengthy mathematical
derivations, an important algorithm in a programming language, input
and output listings, an extract of a standard relating to your thesis,
a user manual, empirical knowledge produced while preparing the
thesis, the results of a survey, lists, pictures, drawings, maps,
complex charts (conceptual schema, circuit diagrams, structure charts)
and so on.


%
% b) The other option is to use numbered chapter and our baseline
% template report.cls numbers them as A, B... The heading and TOC do
% not include prefix 'Appendix' although the page header does.
%\chapter{name of the appendix}
%\label{app:A}                          % For cross-references



\end{document}

