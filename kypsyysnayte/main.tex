\documentclass[12pt,a4paper,finnish
% ,twoside,openright
]{tunithesis}

% Note that you must choose either Finnish or English here and there in this
% file.
% Other options for document class
  % ,twoside,openright   % If printing on both sides (>80 pages)
  % ,twocolumn           % Can be used in lab reports, not in theses

% Ensure the correct Pdf size (not needed in all environments)
\special{papersize=210mm,297mm}


% LaTeX file for BSC/MSc theses and lab reports.
% Requires the class file (=template) tunithesis.cls and figure files,
% either tut-logo, exampleFig (as pdf or eps) and example_code.c
% Author: Lucas Machado (2018)
% Based on TTU template by Sami Paavilainen (2006), modified by Heikki Huttunen (2014)

% More information about Latex basics:
% [Tobias Oetiker, Hubert Partl, Irene Hyna, Elisabeth Schlegl, The
% Not So Short Introduction to LATEX2e, Version 5.03, April 2014, 171
% pages.  Availbale: http://tobi.oetiker.ch/lshort/lshort.pdf]

\author{Väinö-Waltteri Granat}
%\title{Zero to DLA: Building Software Support For Custom RISC-V SoC To Accelerate Deep Neural Networks} % primary title (for front page)
\title{Zero to DLA: Building Software Stack for Accelerating Deep Neural Networks on Custom RISC-V SoC} % primary title (for front page)
\thesistype{Diplomityö} % or Bachelor of Science, Laboratory Report...

% Put your thesis' main language last
% http://mirrors.ctan.org/macros/latex/required/babel/base/babel.pdf
\usepackage{lastpage}
\usepackage[finnish]{babel}
\usepackage[
backend=biber,
style=numeric-comp,
citestyle=numeric-comp,
autocite=footnote,
maxbibnames=5
]{biblatex}
\usepackage{csquotes}
\usepackage{booktabs}
\usepackage{adjustbox}
\usepackage{subcaption}
\usepackage{caption}
\usepackage{svg}
\usepackage[withpage]{acronym}
\usepackage{tikz}
\usetikzlibrary{arrows.meta, arrows}
\usetikzlibrary{matrix, positioning}
\usepackage{xcolor}
\usepackage{amsfonts}
\usepackage{listings, listings-rust}
%\usepackage{inconsolata}
\usepackage{graphicx}
\usepackage{float}
\usepackage{multirow}
\usepackage{calc}

\addbibresource{thesis_refs.bib} %Imports bibliography file
\addbibresource{zotero.bib} %Imports bibliography file

% Table caption on top
\floatstyle{plaintop}
\restylefloat{table}

% Fix this
\newcommand{\fixthis}[1]{\textbf{\textit{\textcolor{red}{[#1]}}}}

\definecolor{tunipurple}{RGB}{78, 0, 142}

\newcommand\todo[1]{{\color{red}!!!TODO: #1}} % Remark text in braces appears in red
\newcommand{\angs}{\textsl{\AA}}              % , e.g. slanted symbol for Ångstöm
\def\checkmark{\tikz\fill[scale=0.4](0,.35) -- (.25,0) -- (1,.7) -- (.25,.15) -- cycle;}
\def\scalecheck{\resizebox{\widthof{\checkmark}*\ratio{\widthof{x}}{\widthof{\normalsize x}}}{!}{\checkmark}}
\lstdefinestyle{customc}{
  belowcaptionskip=1\baselineskip,
  breaklines=true,
  frame=L,
  xleftmargin=\parindent,
  language=C,
  showstringspaces=false,
  basicstyle=\footnotesize\ttfamily,
  keywordstyle=\bfseries\color{green!40!black},
  commentstyle=\itshape\color{purple!40!black},
  identifierstyle=\color{blue},
  stringstyle=\color{orange},
  postbreak=\mbox{\textcolor{red}{$\hookrightarrow$}\space},
}

\lstdefinestyle{customasm}{
  belowcaptionskip=1\baselineskip,
  frame=L,
  xleftmargin=\parindent,
  language=[x86masm]Assembler,
  basicstyle=\footnotesize\ttfamily,
  commentstyle=\itshape\color{purple!40!black},
}

\lstset{escapechar=@,style=customc}

\pagenumbering{roman} % was: {Roman}
\pagestyle{headings}
\begin{document}

% Special trick so that internal macros (denoted with @ in their name)
% can be used outside the cls file (e.g. \@author)
\makeatletter

\thispagestyle{empty}
\vspace*{-.5cm}\noindent

\begin{figure}
    \vspace{-1.3cm}
    \advance\leftskip-2.5cm
    \noindent\includegraphics{img/tunilogo.png}
\end{figure}
 
\vspace{2.5cm}
\begin{flushright}
\noindent\textsf{\LARGE{\@author}}

\noindent\vspace{0.5cm}

\noindent\Huge{\textsf{\textbf{\textcolor{tunipurple}{\@title}}}}
\end{flushright}
\vspace{10.7cm} % adjust to 12.7 this if thesis title needs two lines

% Last some additional info to the bottom-right corner
\begin{flushright}  
    \begin{spacing}{1.0}
      \textsf{Faculty of Information Technology and Communication Sciences (ITC)\\
      \@thesistype\\
      December 2024}
    \end{spacing}
\end{flushright}

% Leave the backside of title page empty in twoside mode
\if@twoside
\clearpage
\fi

%\clearpage
\chapter*{\@aidisclaimertitle}
Artificial intelligence (AI) has been used in generating this work:


I hereby declare, that the AI-based applications used in generating this work are as follows:

\begin{center}
    \begin{tabular}{c|l}
        \toprule
        \textbf{Application} & \textbf{Version} \\
        \midrule
        OpenAi, ChatGPT-4 Turbo & April - November 2024 \\
        Microsoft Copilot. & April - November 2024 \\
        \bottomrule
    \end{tabular}
\end{center}

\section*{Purpose of the use of AI}

Large language models were use as an aid for generating code for some of the Tizk diagrams presented in this thesis.

Explain here \emph{in detail}, for which purpose and how AI was utilized in writing this thesis.

\section*{Parts of this work,  where AI was used}

List here all chapters, sections, subsections, tables, figures and so forth,
that were generated by an AI, or that an AI had a hand in generating.

The following figures were created with a help of AI: \ref{fig:rgb-array}, \ref{fig:activation-functions}, \ref{fig:network-simple}, \ref{fig:network-residual}

\section*{Acknowledgement of risks}

I hereby acknowledge, that as the author of this work, I am fully
responsible for the contents presented in this thesis. This includes
the parts that were generated by an AI, in part or in their entirety. I
therefore also acknowledge my responsibility in the case, where use of
AI has resulted in ethical guidelines being breached.

\clearpage


% Turn off page numbering for the first pages
\pagenumbering{gobble}

\chapter*{Abstract}

\begin{spacing}{1.0}
\noindent \@author: \@title\\
\@thesistype\\
Tampere University\\
Master’s Degree Programme in Signal Processing and Machine Learning\\
December 2024
\end{spacing}
\noindent\rule{12cm}{0.4pt}

\vspace{0.5cm}

% ---------------------------------------
% Abstract and keywords
% ---------------------------------------

\noindent Lorem ipsum~

\noindent\textbf{Keywords:} DLA, Deep-Learning, SoC, Virtual Prototype.

~

\noindent The originality of this thesis has been checked using the Turnitin Originality Check service.


\clearpage

\section*{Lyhenteet}
\begin{acronym}
  \acro{VWW}{Visual Wake Words}
\end{acronym}


% \thispagestyle{empty}
% \listoffigures
% \listoftables
\clearpage


\setcounter{tocdepth}{3}              % How many header level are included
\tableofcontents                      % Create TOC

% The actual text begins here and page numbering changes to 1,2...
% Leave the backside of title empty in twoside mode
\if@twoside
%\newpage
\cleardoublepage
\fi


\renewcommand{\chaptername}{} % This disables the prefix 'Chapter' or
                              % 'Luku' in page headers (in 'twoside'
                              % mode)


\chapter{Kypsyysnäyte}
\label{ch:introduction}
\pagenumbering{arabic}
\setcounter{page}{1} % Start numbering from zero because command
                     % 'chapter*' does page break

In recent years neural network based application have become more and more prominent in our everyday-life. The large driver for this has been the adoption of efficient accelerators in mobile device, that have enabled running neural network applications of mobile devices, such as smartphones.

This interest in neural networks has coincided with the industry's move to heterogeneous System-on-chip solutions being used in consumer and professional devices, to improve computational performance.
More often these companies integrate their accelerators into SoCs, which include CPUs, GPUs, memory and other accelerators and peripherals in one package. Apple and Qualcomm have proved with their SoCs that they can attain desktop like performance in a smaller package than was previously possible. The industry moving towards SoCs has generated new interest in developing open-source SoCs.

The goal of this project was to build software support for the Deep-Learning Accelerator in the upcoming Headsail SoC from SoC Hub using a Renode based virtual prototype as the development platform. The goal was to use this concurrent development approach to have software support ready before the chip had been manufactured.


%
% The bibliography, i.e the list of references
%
\newpage

\printbibliography[title=References]
\addcontentsline{toc}{chapter}{Lähteet}


%
% Appendices are optional. 
% This part is semi-ugly at the moment. Please give feedback if can
% improve it.

\end{document}


% LocalWords:  quantizising
